\section{Analog Front End}

In this section, the complete Analog Front End circuit will be presented. This is composed by four different stages, the Pre-Amp stage, the Filter, a Peak Detector and finally a Comparator. The complete circuit can be seen in Figure \ref{fig:CompleteAFE}.

\begin{figure}[H]
    \centering
    \includegraphics*[scale = 0.45]{Images/CompleteAFE.png}
    \caption{Analog Front End full circuit.}
    \label{fig:CompleteAFE}
\end{figure}

\subsection{Pre-Amp}

\subsubsection{Design}

To design the Pre-Amp stage, the variation of the output voltage of the Electret microphone used is critical to define the total gain of this circuit, so the sensibility of the microphone used is described by the following equation:

\begin{equation}
    Sens_{dBV} = 20 \cdot log_{10} (\frac{Sens_{mv/Pa}}{1000mV/Pa})
    \label{eq:SensdBV}
\end{equation}

So the output voltage in function of the Sound Pressure Level in dB (SPLdB), can be obtained from equation \ref{eq:VoltageSPLdB}.

\begin{equation}
    V_{in} = 10^{\frac{SPLdB-42}{20} \cdot \frac{20e^{-6}}{\sqrt{2}} } \cdot 1000 [mV]
    \label{eq:VoltageSPLdB}
\end{equation}

Resulting in the graphic of Figure \ref{fig:GraphSPLdB}.

\begin{figure}[H]
    \centering
    \includegraphics*[scale = 0.4]{Images/GraphSPLdB.png}
    \caption{Electret microphone voltage in function of SPLdB.}
    \label{fig:GraphSPLdB}
\end{figure}

For this project, the SPLdB obtained by the human voice is the most important, and the gain of the Pre-Amp will depend on this factor, if the gain is too low, the output voltage will also be to low for the desired range, but if this gain is too high, the output voltage for the human voice may reach the supply voltage and so, the signal generated by the voice would be lost after this process, resulting in a simple DC value equal to the supply voltage, and only sounds with a lower SPLdB would have an effect on the remaining of the circuit.

So, considering that the human voice SPLdB is contained in $[60, 70]dB$, the output voltage of the Electret microphone will be $[0.112, 0.355]mV$, respectively.

With this information, the total gain of the Pre-Amp circuit needs to a high value, to avoid using resistors with very high, or very low resistance values, two Inverter circuits connected in series, as shown in figure \ref{fig:PreAmp}.

\begin{figure}[H]
    \centering
    \includegraphics*[scale = 0.5]{Images/PreAmp circuit.png}
    \caption{Pre-Amp circuit.}
    \label{fig:PreAmp}
\end{figure}

The resulting AC component of the output voltage is described by equation \ref{eq:PreAmpEq}.

\begin{equation}
    v_{amp} = G_1 \cdot G_2 \cdot v_{in} = \frac{R_{b1}}{R_{a1}} \cdot \frac{R_{b2}}{R_{a2}} \cdot v_{in} [V]
    \label{eq:PreAmpEq}
\end{equation}

To obtain a maximum value of approximately $1V$, a gain of $2800$ is taken into account for this circuit, so assuming $R_{a1} = R_{a2} = 1k\Omega$, the final values for the remaining two resistors will be, $R_{b1} = 40k\Omega$ and $R_{b2} = 70k\Omega$.

In order to guarantee a DC component of $V_{CC}/2$ in the amplified signal, in a single supply circuit, an DC voltage, generated by the circuit in Figure \ref{fig:VrefVcomp} is added to the positive input node of both OpAmps, and finally, by using two capacitors with high capacitive, the DC component after both stages is $V_{CC}/2$.

\begin{figure}[H]
    \centering
    \includegraphics*[scale = 0.3]{Images/VrefVcomp.png}
    \caption{$V_{ref}$ and $V_{comp}$ circuit.}
    \label{fig:VrefVcomp}
\end{figure}

\subsubsection{Simulation}

