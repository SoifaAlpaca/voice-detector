\section{Implementation}

\subsection{Software integration}
\label{sec:SoftwareInt}
Running on the ESP32 microcontroller, the firmware manages the voice activity detection, system control, and communication tasks, with external platforms for data visualization and interaction. A flowchart representing the code running on the MCU can be seen in Figure \ref{fig:firmwareFlowchart}. 

\begin{figure}[H]
    \centering
    \includegraphics*[scale = 0.5]{Images/FirmawareFlowChart.png}
    \caption{Firmware Flowchart.}
    \label{fig:firmwareFlowchart}
\end{figure}

After validating and successfully testing the user interaction and voice detection code independently, the two were merged into a single firmware following the flowchart shown in Figure \ref{fig:firmwareFlowchart}. This presented unforeseen problems, since the voice detecting code is already pushing the limits of what the Esp32 can do, when merged with code that also uses WiFi, which is a process intensive task, the Esp32 was not able to handle everything together.

To address this, audio processing tasks were pinned to Core 1 while WiFi operations were confined to Core 0, leveraging the dual-core architecture of the ESP32. Additionally, the I2S peripheral, critical for audio data handling, was configured with a higher interrupt priority to ensure timely processing. Despite these optimizations, the system's performance remained inadequate, indicating that the ESP32's computational resources are insufficient for handling both tasks simultaneously at the required level of performance.

\textcolor{red}{METER NA SECCAO CERTA}
Future work should explore the use of a more powerful MCU capable of handling the increased computational and memory demands. Possible candidates could be the ESP32-S3,STM32F746 or Teensy 4.1 .

\subsection{Analog Front End}

To help with the calibration to reach the correct level of gain for the implementation, the second resistor in the Pre-Amp circuit, $R_{b2}$ was changed to a potentiometer, thus, through testing, the correct value for the final gain of the Pre-Amp was reached, which guarantees of the human voice detection in any environment.

\subsubsection{Test}

To test and verify the implemented filter circuit, an analysis, equal to the AC analysis made during the simulations, was carried out, with the help of the ADALM 2000, the resulting frequency response diagram can be seen in Figure \ref{fig:AnalogFilterBodeScoppy}.

\begin{figure}[H]
    \centering
    \includegraphics*[scale = 0.5]{Images/AnalogFilterScoppyBode.png}
    \caption{Implemented Filter Bode Diagram.}
    \label{fig:AnalogFilterBodeScoppy}
\end{figure}

The response above presents the expected behavior, with just a small difference in the first Cutoff Frequency, where the overshoot how appeared before disappeared. This change will not cause any problems for the performance of the Bandpass Filter.

Finally, to test and verify the complete Analog Front End circuit, an oscilloscope was connected to the output of the filter and the comparator. The final results of this test are exposed in Figure \ref{fig:AFETest}.

\begin{figure}[H]
    \centering
    \includegraphics*[scale = 0.5]{Images/AFEtest.png}
    \caption{Analog Front End final test.}
    \label{fig:AFETest}
\end{figure}

Both signals have different scales and reference points, because of this fact, the output of the filter does not appear with the correct values in the graphic above. 

With this information in mind, form this test, it can be concluded that all stages of the Analog Front End worked as expected, from the Pre-Amp, which was able to amplify the signal generated by the human voice in moment $510$ to $525$ to a value high enough to ensure the correct functioning of the rest of the circuit, moving on to the Sixth Order Bandpass Filter, where only the signal expected passed to the final stage of the AFE, expected some disturbances, which did not cause any false wake up signal to be read by the ESP32, since these were not superior to the value of the Comparator circuit. 

\section{Digital Interface}

\textcolor{red}{\textbf{i2s configuration}}
