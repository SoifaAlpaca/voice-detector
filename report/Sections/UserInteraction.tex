\section{User Interaction}

In this section the interaction between the user and the voice activity detecting (VAD) system will be discussed. This component of the project aims to deliver information about the system and provide the user with an interface to change system options. Three primary tools were implemented to achieve this goal: a Graphical User Interface (GUI) for direct interaction, a Grafana dashboard for real-time data visualization, and an additional Home Assistant dashboard designed to extend functionality and integrate with other Internet of Things (IoT) ecosystems. The information exchanged between the MCU and the user is displayed in Table \ref{tab:mcuData} and Table \ref{tab:mcuOptions}.

\begin{table}[H]
    \centering
    \caption{MCU data}
    \begin{tabularx}{\textwidth}{
        >{\centering\arraybackslash}X 
        >{\centering\arraybackslash}X 
        }
        \toprule   
        \textbf{Data Provided} & \textbf{Values}\\
        \midrule
        LED status indicator & True/False  \\
        \midrule
        Speaker Volume & $[0,100]$\\
        \midrule
        MCU State & Idle,Listening,Playing\\
        \midrule
        Time until sleep & $[0,120]$ s\\
        \bottomrule
    \end{tabularx}
    \label{tab:mcuData}
\end{table}

\begin{table}[H]
    \centering
    \caption{System Options}
    \begin{tabularx}{\textwidth}{
        >{\centering\arraybackslash}X 
        >{\centering\arraybackslash}X 
        }
        \toprule  
        \textbf{Option} & \textbf{Values}\\
        \midrule
        LED status indicator & True/False \\
        \midrule
        Speaker Volume & $[0,100]$\\
        \midrule
        Start Sleep & N/A \\
        \midrule
        Time until sleep & $[0,120]$ s\\
        \midrule
        MQTT & On/Off \\
        \midrule
        No Sleep mode & True/False \\
        \bottomrule
    \end{tabularx}
    \label{tab:mcuOptions}
\end{table}

\subsection{Wireless Communication} 

Any wireless communication was handled by MQTT (Message Queuing Telemetry Transport) protocol. MQTT is a lightweight messaging protocol designed for publish/subscribe communication. It operates on a client-server architecture where devices (clients) communicate by publishing messages to specific topics on a central broker. Other clients subscribe to these topics to receive the messages. 

This makes this method of communication suitable for this project since there is no need for low latency communication and makes the program relatively lightweight and ensures flexibility and scalability in IoT networks.

\subsubsection{Grafana}

Grafana is an open-source platform for monitoring and visualizing time-series data. Grafana was integrated as the visualization component of the user interaction system to provide users with a clear and intuitive interface for data analysis. 

It is important to note that not all values are numeric, so in order to display this values to grafana the conversion to numeric values is shown in Table \ref{tab:grafanaVal}.

\begin{table}[H]
    \centering
    \caption{MCU Data in Grafana}
    \begin{tabularx}{\textwidth}{
        >{\centering\arraybackslash}X 
        >{\centering\arraybackslash}X 
        >{\centering\arraybackslash}X 
        }
        \toprule  
        \textbf{Data} & \textbf{Value} & \textbf{Grafana} \\
        \midrule
        LED status indicator & True/False & 0,1 \\
        \midrule
        MCU status & Idle,Listening,Playing & 0,1,2\\
        \bottomrule
    \end{tabularx}
    \label{tab:grafanaVal}
\end{table}

Grafana by itself it is not able to receive and store the MQTT data, therefore there is a need to use Telegraf to translate MQTT messages to Flux and influxDB to store and send the data to Grafana, this architecture is shown in Figure \ref{fig:GrafanaComArch}.

\begin{figure}[H]
    \centering
    \includegraphics*[scale = 0.28]{Images/GrafanaComArch.png}
    \caption{Grafana data flow architecture\textsuperscript{\cite{Coelhomatias_2024}}.}
    \label{fig:GrafanaComArch}
\end{figure}

The final dashboard with real data sent by the Esp32 can be seen in Figure \ref{fig:GrafanaDashboard}.

\begin{figure}[H]
    \centering
    \includegraphics*[scale = 0.3]{Images/GrafanaDashboard.png}
    \caption{Grafana Dashboard.}
    \label{fig:GrafanaDashboard}
\end{figure}

\subsubsection{Home Assistant}

Home Assistant is an open-source home automation platform designed to provide centralized control and monitoring of smart devices within a connected environment. This tool allows advanced automations and integrations between different IoT systems and highly customizable dashboards. This means that in the future with other IoT devices, that are not necessarily made to be used together, can be integrated. 

Two important concepts for Home Assistant are sensors and automations. In Home Assistant, sensors are entities that provide data about the environment or the state of connected devices, this is the data acquisition aspect. Automations, in this context, enable actions to be triggered automatically based on specific conditions, events, schedules or sensor state.

Although not originally specified in the lab work requirements, a Home Assistant integration and a dashboard was developed as part of this project to enhance the system's functionality. This option was considered a better option since grafana is used to analyze and display data over time while Home Assistant allows for deeper automation and integration with other IoT systems, while still allowing visualization of data evolution over time. 

\begin{figure}[H]
    \centering
    \includegraphics*[scale = 0.5]{Images/HADashboard.png}
    \caption{Home Assistant Dashboard.}
    \label{fig:HADashboard}
\end{figure}

The dashboard shown in Figure \ref{fig:HADashboard} is more intuitive and allows both system visualization and control. Another big advantage is the fact that Home Assistant has a smartphone application.

Since this was not part of the labwork requirements, the Home Assistant implementation will be explain in greater detail. 

In the Home Assistant ecosystem most of the dashboard's cards can be build and customized through the graphical interface the more complex configurations is through the yaml configuration files. 

There is only one yaml file that needs to be edited for this project, the configuration.yaml, but in order to make the code easier to read two other files were edited and or created. 

\begin{minted}[frame=lines,fontsize=\small]{yaml}
automation: !include automations.yaml
mqtt: !include mqtt.yaml

input_number:
  esp_volume:
    icon: mdi:volume-medium
    name: Volume
    initial: 30
    min: 1
    max: 100
    step: 1

input_text:
  mqtt_value_input:
    name: MQTT Value Input
    initial: ""
    max: 255

input_boolean:
  led_mode:
    name: LED Mode
    initial: off
\end{minted}

The previous code shows the configuration.yaml, it first includes the two other files that were edited in order to add the VAD functionality and then creates the volume slider, the text input card and a boolean for the LED.

\begin{minted}[frame=lines,fontsize=\small]{yaml}
    - sensor:
        - name: "Status"
          state_topic: "ems/g1/t1/status"
    - sensor:
        - name: "Volume"
          state_topic: "ems/t1/g1/volume/state"
    - sensor:
        - name: "Idle LED"
          state_topic: "ems/t1/g1/enableled/state"
\end{minted}

The mqtt.yaml file creates three sensors, this way Home Assistant is able to receive data from the MCU and use it as sensor.


\begin{minted}[frame=lines,fontsize=\small]{yaml}
- alias: 'Send Volume'
  trigger:
    - platform: state
      entity_id: input_number.esp_volume
  action:
    - service: mqtt.publish
      data:
        topic: ems/t1/g1/volume/set
        payload: "{{ states('input_number.esp_volume') | int }}"
        qos: 0
        retain: false
      target: {}

- alias: 'Validate MQTT Input'
  trigger:
    - platform: state
      entity_id: input_text.mqtt_value_input
  action:
    - choose:
      - conditions:
        - condition: template
          value_template: >
              {{ states('input_text.mqtt_value_input').isdigit() and states('input_text.mqtt_value_input')|int > 0 }}
        sequence:
          - service: mqtt.publish
            data:
              topic: ems/t1/g1/timesleep/set
              payload: "{{ states('input_text.mqtt_value_input') }}"
              qos: 0
              retain: false
            target: {}
      - conditions: []
        sequence:
          - service: notify.persistent_notification
            data:
              message: "Invalid input! Please enter a positive integer."

- alias: "LED On"
  trigger:
  - platform: state
    entity_id: input_boolean.led_mode
    to: "on"
  action:
  - service: mqtt.publish
    data:
      topic: ems/t1/g1/enableled/set
      payload: "1"
    target: {}
- alias: "LED Off"
  trigger:
  - platform: state
    entity_id: input_boolean.led_mode
    to: "off"
  action:
  - service: mqtt.publish
    data:
      topic: ems/t1/g1/enableled/set
      payload: "0"
    target: {}

\end{minted}

The automation.yaml is the more complex file, the first alias detects changes in the volume bar and send a MQTT message to the MCU. The second detects if the input in the text box is a number if it is sends the MQTT message, if not sends a notification saying "Invalid input! Please enter a positive integer.". The rest of the file will ensure the correct value of the LED boolean and according to that boolean will send a MQTT message to turn the activity LED on or off.

As a last example of how integrations can be made, the sleep button is a basic button created through the GUI, it's configurations can be viewed in Figure \ref{fig:SleepCard}.

\begin{figure}[H]
    \centering
    \includegraphics*[scale = 0.3]{Images/SleepCard.png}
    \caption{Sleep card configuration.}
    \label{fig:SleepCard}
\end{figure}

\subsection{Wired Communication}

The GUI was developed to serve as a user-friendly interface for interacting with the system over the wired connection. A wired connection is important to ensure a reliable with low latency and direct interaction to the VAD system. This way the system is still usable when there is no internet connection. The wired communication was all performed through serial communication.

\subsubsection{Graphical User Interface (GUI) }

This GUI provides a user-friendly and interactive platform for controlling and configuring the system. To ensure cross-platform compatibility and ease of development, the GUI was implemented using Python. The interface was designed with a focus on clarity, responsiveness, and functionality.

\begin{figure}[H]
    \centering
    \includegraphics*[scale = 0.3]{Images/GUI.png}
    \caption{Graphical User Interface.}
    \label{fig:GUI}
\end{figure}

As seen in Figure \ref{fig:GUI}, the GUI is divided in five sections.

Serial settings, here the user is able to select the communication port, this allows the use of multiple MCU's and to disconnect and reconnect in another port, which is useful for debugging.

Commands, this section contains four buttons that send a non numeric command, according to Table \ref{tab:mcuOptions}.

Volume Control, a simple volume slider that sends its value when changed.

Sleep Time, a text box that test if the input is valid before sending.

Log, is just a text box that shows information about the system, such as errors in communication.